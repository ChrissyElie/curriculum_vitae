%% start of file `cv.tex'.
%% Myles Borins CV ~ Dec 2012
%%
%% Adapted from moderncv template.tex
%% Copyright 2006-2012 Xavier Danaux (xdanaux@gmail.com).
%
% This work may be distributed and/or modified under the
% conditions of the LaTeX Project Public License version 1.3c,
% available at http://www.latex-project.org/lppl/.


\documentclass[10pt,a4paper,sans]{moderncv}   % possible options include font size ('10pt', '11pt' and '12pt'), paper size ('a4paper', 'letterpaper', 'a5paper', 'legalpaper', 'executivepaper' and 'landscape') and font family ('sans' and 'roman')


% moderncv themes
\moderncvstyle{banking}                        % style options are 'casual' (default), 'classic', 'oldstyle' and 'banking'
\moderncvcolor{black}                          % color options 'blue' (default), 'orange', 'green', 'red', 'purple', 'grey' and 'black'

% adjust the page margins
\usepackage[scale=0.85]{geometry}

% personal data
\firstname{Myles}
\familyname{Borins}
% \address{123 Fake St.}{Stanford, California 94305}

\mobile{+1~(650)~485~3108}
\email{myles.borins@gmail.com}
\homepage{www.mylesborins.com}
\extrainfo{www.github.com/mylesborins}

%----------------------------------------------------------------------------------
%            content
%----------------------------------------------------------------------------------
\begin{document}
%-----       resume       ---------------------------------------------------------
\makecvtitle

\section{Experience}

\cventry
  {March 2020 - Present}
  {Staff Developer Advocate, Open Source Programs Office}
  {Google}
  {New York City, New York}{}
  {In this role I help advise Google teams on open source strategy for various programming languages with a specific focus on Foundations and International Standards. As an individual contributor I contribute to a number of organizations in a leadership capacity including Vice-Chair of the OpenJS Foundation Board. Regular member of the OpenJS Cross Project Council, Vice-President of ECMA International, Co-Chair of TC39, and TSC member of the Node.js project.}

\kern-1em

\cventry
  {October 2019 - March 2020}
  {Staff Developer Advocate, Google Cloud Platform}
  {}
  {}{}
  {}

\kern-1em

\cventry
  {April 2018 - September 2019}
  {Senior Developer Advocate, Google Cloud Platform}
  {}
  {}{}
  {}

\kern-1em

\cventry
  {January 2017 - March 2018}
  {Developer Advocate, Google Cloud Platform}
  {}
  {}{}
  {}

\kern-2.5em

\cventry
  {}
  {}
  {}
  {}{}
  {Collaborator on Node.js project with a focus on governance, stability, and release management. Represent Google as a board member and Vice-Char of the OpenJS Foundation. Collaborate on standards work at TC39. Represent Google as Vice-President of Ecma International. Work closely with various product and engineering teams to ensure our product offerings are desirable to developers and follow industry best practices.}

\cventry
  {October 2015 - January 2017}
  {Node.js Collaborator}
  {IBM}
  {San Francisco, California}{}
  {Full time collaborator on the Node.js project and member of the Core Technical Committee. Specific work includes release management for LTS, writing tooling for smoke testing, maintaining ecosystem stability, and general maintainership for the Node.js project.  Responsibilities include the building and signing of LTS releases for Node.js, currently the most actively downloaded and used versions of the platform. Involvement in a number of Node.js working groups including LTS, Build, HTTP, and Security.  }

%\cventry
%  {April 2014 - September 2015}
%  {Head of Open Source}
%  {Famo.us}
%  {San Francisco, California}{}
%  {Responsible for managing the front line of the Famo.us community. The position was part Product Management, part Developer Relations, part Software Engineering, and a whole bunch of empathy.  Responsible for implementing the companies open source workflow for release management, tooling, and governance. In collaboration with the jQuery Foundation organized jQuerySF in the Spring of 2015, a sold out single track conference with over 1000 attendees.}

%\cventry
%  {September 2013 - December 2013}
%  {Teaching Assistant}
%  {Stanford University}
%  {Stanford, California}{}
%  {Teaching assistant for Music 250a: Physical Interaction Design for Music.  Responsible for teaching students unix fundamentals, introduction to embedded linux, introduction to arduino, introduction to physical computing and basic sound synthesis using tools such as puredata}

%\cventry
%  {July 2013 - September 2013}
%  {Lead Front End Developer}
%  {Djz}
%  {San Francisco, California}{}
%  {Designed new development and deployment toolchain for static Angular web application.  Created web based persistent player for desktop and mobile using open source technologies and various apis.}
%
%\cventry
%  {July 2013 - September 2013}
%  {Future Innovation Research and Development}
%  {Cycling 74'}
%  {San Francisco, California}{}
%  {Super secret future product research and development. I can't tell you what it is, but it is pretty cool!.\newline{}http://www.cycling74.com}

%\cventry
%  {June 2012 - August 2012}
%  {Google Summer of Code 2012 Developer}
%  {Inclusive Design Research Centre}
%  {Toronto, Ontario}{}
%  {Implemented "The Automagic Music Maker" a JavaScript library that offers developers and musicians the ability to generate various types of accessible and responsive instruments in the browser.}

% \cventry
%   {2010--2012}
%   {Research Assistant}
%   {OCAD University}
%   {Toronto, Ontario}{}
%   {Developed code and interaction design frameworks for various government funded research projects including "Body Editing" and "Bio Mapping" with Paula Gardiner, and "Haptics" with Michael Page.}
\section{Open Source and Standards}

\cventry
  {May 2019 - Present}
  {Vice Chair, Board of Directors}
  {OpenJS Foundation}
  {}{}{}
\kern-1.5em
\cventry
  {March 2019 - Present}
  {Board Member, Platinum Director}
  {}
  {}{}{}
\kern-1.5em
\cventry
  {March 2019 - May 2019}
  {Board Member, Community Director}
  {}
  {}{}{}

\cventry
  {September 2017 - March 2019}
  {Board Member, TSC Elected Director}
  {Node.js Foundation}
  {}{}{}
  %{Voted into role by the Technical Steering Committee of the Node.js project. Serve on the Board of Directors of the foundation and act as a liason and representative between the foundation and the technical project.}

\cventry
  {December 2019 - Present}
  {Vice-President}
  {ECMA International}
  {}{}{}
\kern-1.5em
\cventry
  {December 2018 - December 2019}
  {Executive Committee Member}
  {}
  {}{}{}
\kern-1.5em
\cventry
  {February 2020 - Present}
  {TC39 Co-Chair}
  {}
  {}{}{}
\kern-1.5em
\cventry
  {January 2017 - Present}
  {TC39 Delegate}
  {}
  {}{}{}
  %{Represent Google on TC39 - the technical committee responsible for the ECMAScript specification (commonly known as JavaScript). My role has included proposing new language features as well as being involved in developing governance for the committee itself}

\section{Education}

\cventry
  {September 2012 - April 2014}{MA/MST}
  {Stanford University}{Stanford, California}
  {\textit{Masters of Music Science and Technology}}
  {Two year masters program at the \textit{Center for Computer Research in Music and Acoustics}.  Studies included human computer interaction design, digital signal processing, systems design, computer music composition, and site-based installation art. \\\textit{Recipient of the Denning Family Fellowship in Fine Arts }}

\cventry
  {September 2009 - June 2012}
  {BFA}
  {OCAD University}
  {Toronto, Ontario}
  {\textit{Integrated Media with minor in Digital Media Studies}}
  {Graduated on the Dean's honor list. Studies primarily focused on interactive audio/visual site-based installation art.  Other course work included physical computing, computer science, media theory, cultural rhetoric, and rapid-prototyping / manufacturing. \\\textit{Recipient of the OCAD University Medal for Integrated Media }}

% \section{Awards}
% \cventry
%   {}{OCAD University}
%   {OCAD University Medal in Integrated Media}
%   {May 2012}{}
%   {Top accolade given to one student from each department at time of graduation}
% \cventry
%   {}{Stanford Arts Institute}
%   {Denning Family Fellowship in Fine Arts}
%   {March 2012}{}{}
% \cventry
%   {}{OCAD University}
%   {Project 31 Integrated Media Faculty Scholarship}
%   {May 2011}{}{}
% \cventry
%   {}{OCAD University}
%   {DFI Award}
%   {May 2010}{}{}
% \cventry
%   {}{OCAD University}
%   {InterAccess Media Prize}
%   {May 2010}{}{}

% \begin{thebibliography}{9}
%
% \bibitem{linuxaudio}
%   M Borins,
%   \textit{From faust to web audio: Compiling faust to javascript using emscripten},
%   Linux Audio Conference Proceedings,
%   2014.
%
% \bibitem{satellite}
%   E Berdahl, S Salazar, M Borins
%   \textit{Embedded Networking and Hardware-Accelerated Graphics with Satellite CCRMA.},
%   NIME Proceedings,
%   2013.
%
% \bibitem{blackbox}
%   R Michon, M Borins, D Meisenholder
%   \textit{The Black Box.},
%   NIME Proceedings,
%   2013.
%
% \bibitem{ocad}
%   M Page, N Logan, P Harrison, A Vasilliev, M Borins, F Paterson
%   \textit{FedDev Ontario?s ARC Initiatives OCAD University Project \#1 Haptic holography},
%   NIME Proceedings,
%   2012.
%
% \end{thebibliography}

%\section{Skills}
%\cvitem{Programming Languages}{JavaScript, C, C++, Python}
%\cvitem{Web Technologies}{Node.js, npm, HTML5, CSS, various preprocessors for both HTML / CSS, aws}
% \cvitem{Web Frameworks / Libraries}{Angular, Famous, jQuery, browserify}
%\cvitem{Systems}{unix, docker, bash, vim, git, Shell Scripting, make, cmake}
% \cvitem{Prototyping}{3D Printing, CNC-Milling, Laser Cutting, Wood/Metal/Plastics shop experience}
% \cvitem{DSP}{rtaudio, Faust, Matlab, web-audio, Max/MSP, PureData, Chuck}
% \cvitem{Project Contributor}{Node.js, npm, Yeoman, Homebrew, Flocking, OSCeleton, Https Everywhere, monome, faust}
\end{document}
